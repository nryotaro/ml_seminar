\documentclass[dvipdfmx,platex]{beamer}
\usetheme{metropolis}           % Use metropolis theme
\usepackage[deluxe]{otf}% 多書体設定を使う
% \renewcommand{\kanjifamilydefault}{gt}
\title{{\mgfamily 機械学習勉強会 第1回}}
\date{June 13, 2017}
\author{{\mgfamily 中村 遼太郎}}
\institute{}
\begin{document}
\mgfamily
\maketitle
\begin{frame}{Table of contents}
  \setbeamertemplate{section in toc}[sections numbered]
  % \tableofcontents[hideallsubsections]
  \tableofcontents[part=1]
\end{frame}

\begin{frame}[fragile]{{\mgfamily 今日の講義の目標}}
次回以降に学ぶアルゴリズムと例題の概要を知る
\end{frame}
\part{Supervised Learning}
\begin{frame}{パラメトリック法}
  モデル(数式)を仮定し,モデルの最適なパラメタを学習する
\end{frame}
\section{Classification}
\begin{frame}{分類}
  クラスに分類された既存データを元に新規データを分類する
\end{frame}
\section{Perceptron}
\begin{frame}{パーセプトロン, モデル}
  線形モデル$f$を設定する
\end{frame}
\begin{frame}{パーセプトロン, 評価基準}
  誤分類の度合$E$が最小になる$w_i$を求める
\end{frame}
\begin{frame}{ロジスティック回帰, モデル}
  パーセプトロンと同じく線形モデル$f$を設定する
\end{frame}
\begin{frame}{ロジスティック回帰, モデル}
  ただし,$|f|$が大きいほど$t$である確率が高いとする
\end{frame}
\begin{frame}{ロジスティック回帰, モデル}
  ただし,$|f|$が大きいほど$t$である確率が高いとする
\end{frame}
\begin{frame}{ロジスティック回帰, 評価基準}
  訓練データが得られる確率$P$を最大にする$w_i$を求める
\end{frame}
\section{Regression}
\begin{frame}{回帰分析, モデル}
  データが$M$次多項式$f$に従うとする
\end{frame}
\begin{frame}{回帰分析, モデル}
  ただし,$x_n$の観測値$t$は,正規分布の確率密度関数に従い$f(x_n)\pm\sigma$の範囲に散らばる  
\end{frame}
\begin{frame}{回帰分析, 評価基準}
  訓練データの集合が生じる確率$P$を最大にするパラメタを求める
\end{frame}

\end{document}


\begin{frame}[fragile]{{\mgfamily アルゴリズムの分類}}
  求めるモデル(数式)のパラメタを求めるかどうか
  データがパラメタを含むモデル(数式)に従うと仮定するか否かで分かれる
  \begin{itemize}
  \item パラメトリック
  \item ノンパラメトリック
  \end{itemize}
\end{frame}
